\documentclass[t]{beamer}

% --- Theme and Color Setup ---
\usetheme{Madrid}

% Define the custom Orange color
\definecolor{MyOrange}{RGB}{230, 85, 10} 

% Apply the orange color to the structure
\usecolortheme[named=MyOrange]{structure}

% Fix footer colors to match the orange theme
\setbeamercolor{palette primary}{bg=MyOrange,fg=white}
\setbeamercolor{palette secondary}{bg=MyOrange!80!black,fg=white}
\setbeamercolor{palette tertiary}{bg=MyOrange!60!black,fg=white}
\setbeamerfont{caption}{size=\footnotesize}

% Packages
\usepackage[utf8]{inputenc}
\usepackage[spanish]{babel}
\usepackage{hyperref}
\usepackage{tikz} 
\usepackage{booktabs}
\usepackage{natbib}

% --- 1. CONFIGURATION FOR CONTENT SLIDES (Middle Pages) ---
\addtobeamertemplate{frametitle}{}{%
    \begin{tikzpicture}[remember picture,overlay]
        \node[anchor=north east, xshift=-0.3cm, yshift=0cm, inner sep=0pt] at (current page.north east) {
            \includegraphics[height=1cm, keepaspectratio]{logo_UPB.jpg}
        };
    \end{tikzpicture}%
}

% --- 2. CONFIGURATION FOR FIRST & LAST PAGE ---
\newcommand{\FirstLastPageLayout}{
    \begin{tikzpicture}[remember picture,overlay]
        \node[anchor=north east, xshift=-0.3cm, yshift=0cm, inner sep=0pt] at (current page.north east) {
            \includegraphics[height=1cm, keepaspectratio]{logo_UPB.jpg}
        };
        \node[anchor=south west, xshift=0.5cm, yshift=0.5cm] at (current page.south west) {
            \textcolor{MyOrange}{\footnotesize Universidad Pontificia Bolivariana}
        };
    \end{tikzpicture}
}

% --- Info ---
\title[Historia IEE]{Historia de la Ingeniería Eléctrica y Electrónica}
\subtitle{De los fundamentos a la era digital}
\author{Curso de Introducción a la IEE}
\institute[UPB]{Universidad Pontificia Bolivariana}
\date{\today}

\begin{document}

% --- Slide 1: Title Page ---
\begin{frame}
    \FirstLastPageLayout
    \titlepage
\end{frame}

% --- Slide 2: ¿Qué es ingeniería? ---
\begin{frame}{¿Qué es ingeniería?}
    \begin{itemize}
        \item La ingeniería es la disciplina que \textbf{diseña, construye y opera soluciones} usando ciencia, matemáticas y experiencia
        \item Considera: \textbf{seguridad, costo, impacto social y confiabilidad}
        \item A medida que las sociedades crecen, la ingeniería pasa de ``obras aisladas'' a \textbf{sistemas}
        \begin{itemize}
            \item Redes de agua, energía, comunicaciones
            \item Un fallo puede afectar a millones de personas
        \end{itemize}
    \end{itemize}
\end{frame}

% ===== HISTORIA GENERAL DE LA INGENIERÍA =====

% --- Slide 3: Ingeniería temprana ---
\begin{frame}{Ingeniería temprana ($\approx$ 10.000 a.C. -- 500 a.C.)}
    \begin{itemize}
        \item \textbf{Revolución agrícola ($\approx$ 10.000 a.C.):}
        \begin{itemize}
            \item Aparición de riego, almacenamiento y planificación del territorio
            \item Ingeniería como \textbf{gestión de recursos}
        \end{itemize}
        
        \item \textbf{Mesopotamia y Egipto ($\approx$ 3500--1500 a.C.):}
        \begin{itemize}
            \item Canales, diques, medición de tierras
            \item Construcción masiva: obra + logística + mantenimiento
        \end{itemize}
        
        \item \textbf{China, el Indo y Mesoamérica:}
        \begin{itemize}
            \item Consolidación de hidráulica, urbanismo y metalurgia
        \end{itemize}
    \end{itemize}
\end{frame}

% --- Slide 4: Mundo clásico ---
\begin{frame}{Mundo clásico e ``ingeniería pública'' ($\approx$ 500 a.C. -- 500 d.C.)}
    \begin{itemize}
        \item \textbf{Grecia:}
        \begin{itemize}
            \item Aporte de matemáticas y mecánica como lenguaje para modelar
        \end{itemize}
        
        \item \textbf{Roma:}
        \begin{itemize}
            \item Escala la ingeniería a infraestructura masiva
            \item \textbf{Caminos, puentes, acueductos}
            \item \textbf{Impacto social:} comercio, salud pública, expansión urbana
        \end{itemize}
    \end{itemize}
\end{frame}

% --- Slide 5: De técnica artesanal a ciencia ---
\begin{frame}{De técnica artesanal a ciencia y profesión ($\approx$ 1500--1900)}
    \begin{itemize}
        \item \textbf{Renacimiento y Revolución Científica:}
        \begin{itemize}
            \item Medición y experimentación sistemática
        \end{itemize}
        
        \item \textbf{Revolución Industrial ($\approx$ 1700--1900):}
        \begin{itemize}
            \item Fábricas, transporte y estandarización
            \item Nace la ingeniería moderna como profesión
        \end{itemize}
        
        \item \textbf{Nueva realidad:}
        \begin{itemize}
            \item Más potencia → más productividad
            \item Más riesgo → necesidad de normas e inspección
        \end{itemize}
    \end{itemize}
\end{frame}

% ===== HISTORIA DE LA INGENIERÍA ELÉCTRICA Y ELECTRÓNICA =====

% --- Slide 6: Primeros fundamentos ---
\begin{frame}{Primeros fundamentos de la electricidad (1600--1831)}
    \begin{itemize}
        \item \textbf{1600 -- William Gilbert:} sistematiza magnetismo y electricidad
        
        \item \textbf{1752 -- Benjamin Franklin:} 
        \begin{itemize}
            \item Experimentos con electricidad y \textbf{pararrayos}
            \item Primer gran hito de \textbf{seguridad eléctrica} con impacto social
        \end{itemize}
        
        \item \textbf{1800 -- Alessandro Volta:}
        \begin{itemize}
            \item \textbf{Pila voltaica}: fuente controlable de corriente continua
        \end{itemize}
        
        \item \textbf{1831 -- Michael Faraday:}
        \begin{itemize}
            \item \textbf{Inducción electromagnética}
            \item Principio de generadores y motores
        \end{itemize}
    \end{itemize}
\end{frame}

% --- Slide 7: Maxwell ---
\begin{frame}{La teoría que habilita la ingeniería (1864)}
    \begin{itemize}
        \item \textbf{1864 -- James Clerk Maxwell:}
        \begin{itemize}
            \item Ecuaciones del electromagnetismo
            \item Unifica electricidad, magnetismo y luz
        \end{itemize}
        
        \item \textbf{Impacto técnico:}
        \begin{itemize}
            \item Permite diseñar con modelos, no solo por prueba y error
        \end{itemize}
        
        \item \textbf{Impacto social:}
        \begin{itemize}
            \item Acelera telecomunicaciones, energía y tecnología industrial
        \end{itemize}
    \end{itemize}
\end{frame}

% --- Slide 8: Guerra de corrientes ---
\begin{frame}{Electrificación y ``guerra de corrientes'' (1879--1890s)}
    \begin{itemize}
        \item \textbf{1879--1882 -- Thomas Edison:}
        \begin{itemize}
            \item Iluminación y sistemas urbanos de \textbf{corriente continua (DC)}
        \end{itemize}
        
        \item \textbf{1880s--1890s -- Nikola Tesla / George Westinghouse:}
        \begin{itemize}
            \item Sistemas de \textbf{corriente alterna (AC)}
            \item Transmisión eficiente con transformadores
        \end{itemize}
        
        \item \textbf{Impacto social:}
        \begin{itemize}
            \item Iluminación masiva y electrificación de industrias
            \item Ciudades nocturnas y nuevos empleos técnicos
        \end{itemize}
    \end{itemize}
\end{frame}

% --- Slide 9: Redes eléctricas ---
\begin{frame}{Redes eléctricas: nace el ``sistema'' de potencia ($\approx$ 1890--hoy)}
    \begin{itemize}
        \item Con la red eléctrica aparece un concepto central:
        \begin{itemize}
            \item \textbf{Generación -- Transmisión -- Distribución -- Protección}
            \item Sistema interdependiente y complejo
        \end{itemize}
        
        \item \textbf{Impacto social:}
        \begin{itemize}
            \item La electricidad se vuelve infraestructura crítica
            \item Afecta: salud, transporte, comunicaciones, economía
        \end{itemize}
    \end{itemize}
\end{frame}

% --- Slide 10: Comunicaciones ---
\begin{frame}{Comunicaciones: del telégrafo a la radio ($\approx$ 1830--1930)}
    \begin{itemize}
        \item \textbf{Siglo XIX -- Telégrafo:}
        \begin{itemize}
            \item Redes de señal a larga distancia
        \end{itemize}
        
        \item \textbf{Primer tercio del siglo XX -- Radio:}
        \begin{itemize}
            \item Comunicación inalámbrica masiva
        \end{itemize}
        
        \item \textbf{Laboratorios industriales influyentes:}
        \begin{itemize}
            \item Ejemplo: \textbf{Bell Labs}
            \item Convierten teoría en tecnología escalable
        \end{itemize}
        
        \item \textbf{Impacto social:}
        \begin{itemize}
            \item Información en tiempo real y coordinación global
        \end{itemize}
    \end{itemize}
\end{frame}

% --- Slide 11: Nace la electrónica ---
\begin{frame}{Nace la electrónica: controlar señales (1904--1950)}
    \begin{itemize}
        \item \textbf{1904 -- John A. Fleming:}
        \begin{itemize}
            \item Diodo de vacío (rectificación)
        \end{itemize}
        
        \item \textbf{1906 -- Lee de Forest:}
        \begin{itemize}
            \item Triodo (amplificación)
        \end{itemize}
        
        \item \textbf{Impacto técnico:}
        \begin{itemize}
            \item Amplificar y procesar señales (audio, radio, telefonía)
        \end{itemize}
        
        \item \textbf{Impacto social:}
        \begin{itemize}
            \item Medios masivos y telecomunicaciones modernas
        \end{itemize}
    \end{itemize}
\end{frame}

% --- Slide 12: Semiconductor ---
\begin{frame}{Semiconductor, miniaturización y era digital (1947--1971)}
    \begin{itemize}
        \item \textbf{1947 -- Transistor:}
        \begin{itemize}
            \item Bardeen, Brattain, Shockley (Bell Labs)
        \end{itemize}
        
        \item \textbf{1958--1959 -- Circuito integrado:}
        \begin{itemize}
            \item Jack Kilby / Robert Noyce
        \end{itemize}
        
        \item \textbf{1971 -- Microprocesador:}
        \begin{itemize}
            \item Computación en un chip
        \end{itemize}
        
        \item \textbf{Impacto social:}
        \begin{itemize}
            \item Automatización, computación personal
            \item Control industrial, medicina moderna
            \item Telecomunicaciones digitales
        \end{itemize}
    \end{itemize}
\end{frame}

% --- Slide 13: Seguridad eléctrica ---
\begin{frame}{Seguridad eléctrica/electrónica (siglo XX--hoy)}
    \begin{block}{La ingeniería madura cuando integra protección + normas + cultura operativa:}
    \end{block}
    
    \begin{itemize}
        \item \textbf{Puesta a tierra:} reduce tensiones peligrosas en fallas
        \item \textbf{Fusibles e interruptores:} limitan sobrecorriente
        \item \textbf{Interruptor diferencial (RCD/GFCI):} detecta fuga a tierra
        \item \textbf{Relés de protección:} aislan fallas, evitan cascadas
        \item \textbf{Normas y estandarización:} IEC, IEEE
        \item \textbf{Prácticas seguras:} LOTO, EPP, análisis de riesgo
    \end{itemize}
    
    \vspace{0.2cm}
    \small{\textit{Muchos ``desastres evitados'' no son noticias; son consecuencia de protecciones y normas que impiden errores repetidos.}}
\end{frame}

% ===== FALLOS EMBLEMÁTICOS =====

% --- Slide 14: Fallos emblemáticos (infraestructura) ---
\begin{frame}{Fallos emblemáticos: infraestructura y sistemas}
    \begin{itemize}
        \item \textbf{1940 -- Puente Tacoma Narrows (EE. UU.):}
        \begin{itemize}
            \item Colapso por vibración inducida por viento
            \item \textit{Lección:} diseñar comportamiento dinámico, no solo resistencia
        \end{itemize}
        
        \item \textbf{1977 -- Apagón de Nueva York:}
        \begin{itemize}
            \item Falla + eventos encadenados → pérdida masiva de suministro
            \item \textit{Lección:} coordinación de protecciones y resiliencia
        \end{itemize}
        
        \item \textbf{2003 -- Apagón del noreste EE. UU./Canadá:}
        \begin{itemize}
            \item Monitoreo deficiente + condiciones operativas → cascada
            \item \textit{Lección:} observabilidad, automatización, mantenimiento
        \end{itemize}
    \end{itemize}
\end{frame}

% --- Slide 15: Fallos emblemáticos (alto riesgo) ---
\begin{frame}{Fallos emblemáticos: alto riesgo (espacio/nuclear/software)}
    \begin{itemize}
        \item \textbf{1967 -- Incendio Apollo 1:} diseño de prueba y materiales
        
        \item \textbf{1975--1987 -- Therac-25:} sobredosis por fallos de software
        
        \item \textbf{1986 -- Chernóbil:} prueba insegura + diseño deficiente
        
        \item \textbf{1986 -- Challenger:} falla de junta + decisiones bajo presión
        
        \item \textbf{1996 -- Ariane 5 Vuelo 501:} error de software no revalidado
        
        \item \textbf{2011 -- Fukushima Daiichi:} evento extremo supera defensas
    \end{itemize}
    
    \vspace{0.2cm}
    \small{\textit{Lección común: cultura de seguridad, redundancia, validación y comunicación técnica.}}
\end{frame}

% --- Slide 16: Cierre ---
\begin{frame}{Lecciones para Ing. Eléctrica y Electrónica}
    \begin{enumerate}
        \item La historia avanza cuando pasamos de ``hacer funcionar'' a \textbf{hacer funcionar de forma segura y repetible}
        
        \item En eléctrica y electrónica, el riesgo se controla con:
        \begin{itemize}
            \item Protecciones (breaker/diferencial/tierra/relés)
            \item Normas, pruebas y procedimientos
        \end{itemize}
        
        \item El impacto social es doble:
        \begin{itemize}
            \item Habilitamos infraestructura (energía y comunicación)
            \item Debemos evitar que su falla produzca daño masivo
        \end{itemize}
        
        \item \textbf{Mensaje final:} la seguridad es parte del diseño, no una etapa posterior
    \end{enumerate}
\end{frame}

% --- Final page ---
\begin{frame}[c]
    \FirstLastPageLayout
    \centering
    \Huge \textcolor{MyOrange}{Gracias por su atención}
    
    \vspace{1.5cm}
    \normalsize
    \textbf{Curso de Introducción a la IEE} \\
    Universidad Pontificia Bolivariana
\end{frame}

\end{document}
